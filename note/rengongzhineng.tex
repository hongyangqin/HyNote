\documentclass[UTF8,a4paper]{ctexart}
\usepackage[margin=1in]{geometry}
\usepackage{fancyhdr,hyperref}
\pagestyle{fancy}
\hypersetup{hidelinks}

\lhead{\bfseries \leftmark}
\chead{}
\rhead{SCUT}
\lfoot{\url{https://github.com/285571052}}
\cfoot{qhy}
\rfoot{\thepage}
\setlength{\headheight}{13pt}
\renewcommand{\headrulewidth}{0.4pt}
\renewcommand{\footrulewidth}{0.4pt}

\setlength{\parindent}{0pt}
\newcommand{\spaceline}{\vspace{\baselineskip}}

\author{ qhy }
\date{\today}
\title{人工智能}

\begin{document}
  \maketitle
  \tableofcontents
  \newpage

  \section{介绍}
  1956年提出人工智能的概念。

  \section{无题}
  \textbf{对于非结构化问题,即不能使用数学来进行描述的问题,我们怎么使用计算机进行描述?}\\
  使用\textbf{状态空间法:}
  \begin{itemize}
    \item 使用多元组来描述问题的状态
    \item 每个问题有初始态和目标抬
    \item 有些状态是非法的
    \item 我们需要做的是,从初始态转移到目标态
  \end{itemize}

\end{document}
