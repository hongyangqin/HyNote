\documentclass[UTF8,a4paper]{ctexart}
\usepackage[margin=1in]{geometry}
\usepackage{fancyhdr,hyperref,amsmath}
\pagestyle{fancy}
\hypersetup{hidelinks}

\lhead{\bfseries \leftmark}
\chead{}
\rhead{SCUT}
\lfoot{\url{https://github.com/285571052}}
\cfoot{qhy}
\rfoot{\thepage}
\setlength{\headheight}{13pt}
\renewcommand{\headrulewidth}{0.4pt}
\renewcommand{\footrulewidth}{0.4pt}

\setlength{\parindent}{0pt}
\newcommand{\spaceline}{\vspace{\baselineskip}}

\author{ qhy }
\date{\today}
\title{编译原理}

\begin{document}
  \maketitle
  \tableofcontents
  \newpage

  \section{介绍}
  2017-9-5:第一节课主要简单介绍的mac地址,ip以及协议相关内容。

  \section{数字调制和复用}
  \textbf{波特率:}每秒信号变化的次数。
  \[C = B \times \log_2 n\]

  \spaceline
  $\left \{ \begin{array}{l}
  \text{基带传输:使用高低调位为标准描述0/1两个信号}\\
  \text{通带传输:使用振幅,频率,相位等为标准描述}
  \end{array} \right .$

  基带传输的几个例子:???
  通带传输的3个例子:???

  信号星座?\\
  通带传输3种情况的综合应用:使用不同相位、波长表示不用的信号,这样接受一个符号,这个符号能表示的范围就变大,从而减少传输的时间。\\
  比如每次接受一个波形,能可能表示4个情况,那么这个波形就能表示两个位(00,01,10,11)的一种

  例子????

  \textbf{复用技术:}多个用户使用同一个通道。\\
  前面相位、波长用来表示数字,而频率和其他冗余信息则主要用于复用技术方面。

  复用技术主要有以下几种类型:
  \begin{itemize}
    \item 频分多路复用(FDM)\\
    不同频率直接迭代,最终经过滤波器分开
    \item 正交FDM\\
    普通的FDM每个频率段之间是不交的,而正交FDM有相交部分,但是也能区分开来
    \item 时分多路复用\\
    时间上共享,一个接一个使用
    \item 统计时分多路复用\\
    时分多路复用的话,是每个用户一次分配到使用时间,不管有没用到,而统计时分多路复用则是没有用到就不分配时间
    \item 码分多路复用\\
    每个用户拥有一个唯一的码片,每个码片相互正交(主要用于3G网络)\\
    比如,传送4位信号,里面能包含3个用户发送的信息情况等。
  \end{itemize}

  冲突:同时发送数据,引起冲突

  怎么防止冲突?减小冲突域

  调制解调器的任务是把数字信号转换成模拟信号

  TCM:每次采样中,有一位用于纠错

  电话线拨号上网(不经过猫)的56K是怎么计算出来的?

\end{document}
