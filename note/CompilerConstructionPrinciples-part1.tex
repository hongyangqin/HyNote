\section{介绍}

\textbf{预期收货}:
\begin{itemize}
 \item 通过学习编译原理,写出更高效的代码
 \item 针对目标,自编写编译器

       对某个模型机的编译器进行设计
\end{itemize}

\textbf{编译程序:}将高级语言书写的程序翻译成等价的低级语言的程序

\textbf{源程序:}编译程序的输入对象

\textbf{目标程序:}编译程序的输入对象

\textbf{编译器和解释器的区别?}
\begin{itemize}
 \item 编译器是转换器,解释器是执行系统。
 \item 解释器是源程序的一个执行系统,工作结果得到源程序的执行结果
 \item 编译器是源程序的转换系统,工作结果得到等价于源程序的某种目标程序
\end{itemize}

\begin{figure}[H]
 \centering
 \includegraphics[scale = 0.3]{assets/CompilerConstructionPrinciples-part1_8c8c5.png}
 \caption{编译器和解释器}
\end{figure}

\spaceline

程序编译的过程主要分为两大阶段:
\begin{itemize}
 \item [1.] 查错
       \begin{itemize}
        \item 词法分析
        \item 语法分析
        \item 语义分析
       \end{itemize}

 \item [2.] 综合(翻译)
       \begin{itemize}
        \item [1.] 中间代码生成
        \item [2.] 代码优化
        \item [3.] 目标代码生成
       \end{itemize}
\end{itemize}

高级程序处理过程:初始源程序$\to$预处理$\to$源程序$\to$编译$\to$目标汇编$\to$机器代码
\begin{figure}[H]
 \centering
 \includegraphics[scale = 0.3]{assets/CompilerConstructionPrinciples-part1_41d77.png}
 \caption{高级语言程序的处理过程}
\end{figure}

注:生成机器代码的时候,并不是直接生成,而是先生成对应的汇编代码,再生成机器代码。

编译过程:每个阶段的输出作为下一个阶段的输入(即数据从一种形式转换成另一种形式)。

编译过程的每个阶段都包括两个相同的处理:表格管理和出错处理。

表格管理是保存编译过程每个阶段的数据和结果,出错处理则是对编译过程遇到的语法,词法等错误进行处理。

\begin{figure}[H]
 \centering
 \includegraphics[scale = 0.3]{assets/CompilerConstructionPrinciples-part1_95806.png}
 \caption{编译的各个阶段}
\end{figure}

\subsection{词法分析}
词法分析的任务:从左到右一个字符一个字符地读入源程序,对构成源程序的字符流进行扫描和分解,从而识别出一个个单词(\textbf{token})

主要分为两大步骤,扫描和分解

扫描为从左到右扫描

分解为以界符对语句进行分割(注:双引号内的介符不可划分)

最后结果使用一个二元组保存,格式为 \textbf{(种类,值)}

\subsection{语义分析}
语义分析:不断构造语法树

如:类型不对,数组越界等

过程:单词符号串$\to$语法分析$\to$语法短语

识别规则:描述程序结果的规则,通常由递归规则表示。

输出:合法的语法树

\subsection{语义分析}
任务:审查源程序是否有语义错误,为代码生成阶段收集类型信息

语义分析:包括静态语义和动态语义

常见的错误包括类型不匹配,数组越界等

输出:生成中间代码(把中间代码生成合并到语义分析部分,拆分开也可以)

结果使用四元式表示,格式为:(运算符,运算对象1,运算对象2,结果)

\textbf{静态语义:}
\begin{itemize}
 \item 上下文相关性
 \item 类型匹配:每个算符是否具有语言规范允许的运算
 \item 类型转换
\end{itemize}

\subsection{中间代码生成 intermediate code generation}
任务:将源程序生成一种内部表示形式,这种内部表示形式叫 \textbf{中间代码}

一般使用四元式表示,格式为:(运算符,运算对象1,运算对象2,结果)

\subsection{代码优化}
任务:对中间代码进行等价变换,以便生成更高效的目标代码。

\subsection{目标代码生成}
任务:把中间代码转换成特定机器上的绝对指令代码或可重定位的指令代码或汇编指令代码,它的工作与硬件系统和指令含义有关

目标代码生成:与目标机器紧紧相关,先生成对应的汇编代码,再生成机器代码

\subsection{表格管理与出错处理}
表格管理与出错处理:每个阶段都执行这一操作

\textbf{表格管理}:编译过程中源程序的各种信息被保存在种种不同的表格里,编译各阶段的工作涉及到构造、查找或更新有关的表格,因此需要有表格管理工作。

\textbf{出错处理}:编译过程中,发现源程序有错误(词法错误,语法错误,语义错误),编译程序应报告错误的性质和错误的地点,并将错误所造成的影响限制在尽可能小的范围内,使得源程序的其余部分继续被编译下去。这些工作称为出错处理(\textbf{error handling})。

\subsection{编译程序划分}
编译程序划分:分析与综合(翻译)两个阶段

按是否与目标机器相关:前端(无关)和后端(相关)

\section{文法和语言}

掌握自下而上和自上而下的分析方法。

\spaceline
\textbf{程序设计的定义:}语言是一个记号系统。

\textbf{分析程序设计语言的两个阶段:}$\left \{\begin{array}{l} \text{每个程序的构成规律}\\\text{每个程序的含义}\end{array} \right .$

\subsection{语法}
\textbf{语法:}是一组规定,用它可以形成和产生一个合适的程序。

描述工具:文法

语法只可以判断结构是否合法。

\subsection{语义}
\textbf{语义:}$\left \{ \begin{array}{l}\text{静态语义}\\\text{动态语义} \end{array} \right.$

\spaceline

\textbf{静态语义:}一系列限定的规则,确定哪些合乎语法的程序是合适的。

\textbf{动态语义:}运行或动态链接时进行的判断。

描述工具:指称语义,操作语义

作用:检查类型匹配,变量作用域等。

\textbf{静态语义:}
\begin{itemize}
 \item 上下文相关性
 \item 类型匹配:每个算符是否具有语言规范允许的运算
 \item 类型转换
\end{itemize}

\subsection{文法}
\textbf{如何描述语句?}
\begin{itemize}
 \item [1.] 生成方式(文法)\\
       语言中每个句子可以用严格定义的规则进行构造
 \item [2.] 识别方式(自动机)\\
       用一个过程,经有限次计算后会停止回答"是" ,若属于句子,要么回答"否",要么永远持续下去
\end{itemize}

\spaceline
\textbf{语言:}可以是有穷的也可以是无穷的,我们要做的是,找出语言的有穷表示。

\textbf{文法的作用:}
\begin{itemize}
 \item [1.] 使用有穷的规则描述无穷的语言
 \item [2.] 严格定义句子的结构,是判断句子结构是否合法的依据
\end{itemize}

\textbf{方法:}\\
\hl{::=}表示定义一条规则\\
\hl{$\Rightarrow$}表示应用这条规则,完成句子的变换(即由...推导出...的意思)

\subsubsection{符号和符号串}
\textbf{字符表(符号集,字母表):}由字母、数字和若干专用字符组成的非空有限集合。

\textbf{符号串:}字母表中的符号组成的任何有穷序列。

比如:$a,aca$是$A:\{a,b,c\}$的符号串。

\textbf{空串}:不含任何符号的符号串 , 用$\epsilon$ 表示

符号串的长度:符号串含有符号的个数,$|\epsilon| = 0$

符号串的运用:
\begin{itemize}
 \item 连接\\
       定义:$x = "ST", y = "aby"$\\
       则$xy = "STaby"$

 \item 方幂\\
       $a^n = \begin{array}{c}\underbrace{aa\cdots aa}\\ \text{n个a} \end{array}$

       注:$a^0 = \epsilon$

 \item 集合的乘积\\
       定义:$A = \{a,b\} , B = \{0 ,1\}$\\
       则$AB = \{a0 , a1 , b0 , b1\}$\\
       注:A,B是符号串的集合,并且AB的结果中A的符号串在前面

 \item 集合的方幂\\
       $A^2 = AA$
\end{itemize}

\subsubsection{闭包}
\textbf{闭包:}$\Sigma$的闭包为$\Sigma$上所有元素组合的集合$\Sigma^*$。即
\begin{equation}
 \Sigma^* = \Sigma^0 \bigcup \Sigma^1 \bigcup \Sigma^2 \cdots
\end{equation}

\textbf{正闭包}:闭包去掉空集元素。即
\begin{equation}
 \Sigma^+ = \Sigma^* - \{\epsilon\}
\end{equation}

有:
\begin{equation}
 \Sigma^+ = \Sigma \Sigma^* = \Sigma^* \Sigma
\end{equation}

字母表上的一个语言是$\Sigma$上的一些符号串的集合,即该语言是$\Sigma^*$的一个子集

注:\textbf{$\{\epsilon\}$也是一个语言 } ,即 $\Phi$(空集)是一个语言

\subsubsection{文法}
\textbf{产生式(规则):}一组有序对$(\alpha , \beta)$ 表示$\alpha \to \beta$ 或$\alpha ::= \beta$

\textbf{文法的定义:}四元组$(V_N,V_T , P , S)$

其中,$V_N$表示非终结符 , $V_T$表示终结符 , $P$表示产生式集合,$S$表示文档起始符号。

注:
\begin{itemize}
 \item $S$为文档的其实符号,从具体情况上看,我们就是从S起始的产生式对句子进行解析。
 \item $V_N \bigcup V_T = \text{字母表}$
 \item S是一个非终结符,并且至少在一条产生式的左部出现
\end{itemize}

例子:

文法$G = (V_N,V_T , P , S)$\\
$V_N = \{S\}, P = \{S \to 0S1 , S\to 01\}, V_T = \{0,1\}$\\
开始符号为 $S$

\spaceline
文法$G = (V_N,V_T , P , S)$\\
$V_N = \{\text{标识符,字母,数字}\}$\\
$ V_T = \{a,b,\cdots,z,0,1,\cdots , 9\}$\\
$P = \left \{
 \begin{array}{l}
  <\text{标识符}> \to <\text{字母}>,                \\
  <\text{标识符}> \to <\text{标识符}><\text{字母}>, \\
  <\text{标识符}> \to <\text{标识符}><\text{数字}>, \\
  <\text{字母}> \to a,                              \\
  \cdots ,                                          \\
  z,<\text{字母}>                                   \\
  <\text{数字}> \to 0,                              \\
  \cdots ,                                          \\
  9,<\text{数字}>
 \end{array}\right \}$\\
$S = <\text{\text{标识符}}>$

\subsubsection{简化表示}
简化表示:只使用产生式来表示文法,四元组的其他三元在产生式中表示。
\begin{itemize}
 \item 第一条产生式的左部表示$S$
 \item 用大写字母或者尖括号包围表示非终结符集合
 \item 用小写字母表示终结字符集合
 \item 左部相同的产生式的多个,可以用\hl{$|$}(或) 来简化表示。\\
       比如:$A\to \alpha , A \to \beta$可以简记为$A \to \alpha | \beta$\\
       注意:$A \to \alpha | \beta$表示的是两条产生式而不是一条,其中$\alpha , \beta$为候选式。
\end{itemize}

\subsection{归纳与推导}
\textbf{直接归纳与直接推导:}\\
若$v\Rightarrow w$则称$v$直接推导$w$ 或 $w$ 直接归纳为 $v$

\spaceline

\textbf{归纳与推导:}使用了多条产生式,记为
\begin{equation}
 \renewcommand{\arraystretch}{0.5}
 S \begin{array}{c} + \\ \Rightarrow \end{array} \alpha
\end{equation}

若$S  \Rightarrow^+ \alpha$ 或$S = \alpha$则记作:
\begin{equation}
 \renewcommand{\arraystretch}{0.5}
 S \begin{array}{c} * \\ \Rightarrow \end{array} \alpha
\end{equation}


\sout{注:如果使用了一条产生式,进行了多次$\Rightarrow$也为直接推导, 推导是使用了多条产生式。}推导指的是多次直接推导,无论是否使用同一条产生式

\textbf{直接推导:}就是用产生式的右部替换产生式的左部的过程

\textbf{直接归约:}就是用产生式的左部替换产生式的右部的过程

\subsection{句型、句子、语言}
\textbf{句型与句子的定义:}\\
对于文法$G[S]$,若$\renewcommand{\arraystretch}{0.5}S \begin{array}{c} * \\ \Rightarrow \end{array} x$ ,则称$x$为文法$G$的句型。\\
若$x$仅由终结符组成,则称$x$为文法$G$的句子

注:
\begin{itemize}
 \item 句型可以表示多个句子,句子是句型的一个情况之一。句子也是一个句型。
 \item 起始符也为句型,即\\
       隐含条件:$S\to S$ , 所以$S$也是文法$G$的句型。
\end{itemize}

\spaceline
\textbf{语言$L[G]$的定义:}语言$L[G]$是文法$G[S]$的所有句子的集合。

注:
\begin{itemize}
 \item $+,(,*,$等出现在句子中的成分,也是终结符。
 \item 可以通过产生式的具体表达,判断出某个操作的优先级。
\end{itemize}

\spaceline
\textbf{文法的等价:}\\
若$L[G_1] == L[G_2]$ ,则称文法$G_1$ 和 文法$G_2$ 等价。

\spaceline
\textbf{文法的种类:}
\begin{itemize}
 \item 0型文法(短语文法)\\
       对任一产生式$\alpha \to \beta$ , 都有$\alpha \in (V_N\cup V_T)$ ,且至少含有一个终结符,$\beta \in (V_N\cup V_T)$\\
       即 产生式左侧只要不是句子,就是0型文法\\
       对产生式基本无限制
 \item 1型文法(上下文相关文法)\\
       在0型文法条件下,对任一产生式$\alpha \to \beta$ ,都有$|\beta| \leq |\alpha|$ , 仅仅$S \to \epsilon $除外($|\alpha|$指的是句型的长度)\\
       即产生式左侧短于右侧,就是1型文法\\
       上下文体现在是对产生式左部的扩展,拆分成多个部分,当扩展其中一个部分的时候,就需要考虑其他部分(上下文)\\
       $\alpha A \gamma \to \alpha \beta \gamma , \gamma \in V^* , A\in V_N , \beta \in V^+$ , 将$A$替换成$\beta$时,必须考虑$A$的上下文$\alpha , \gamma$
 \item 2型文法(上下文无关文法)\\
       在1型文法条件下,对任一产生式$\alpha \to \beta$ ,都有$\alpha \in (V_N) , \beta \in (V_N\cup V_T)$\\
       即产生式左侧只有非终结符\\
       \textbf{大部分程序设计语言是2型文法}

 \item 3型文法(正规文法)\\
       在2型文法的条件下,,对任一产生式$\alpha \to \beta$,都有形如$A\to aB$或$A \to a$,其中$A,B\in V_N , a \in V_T$\\
       即产生式的右侧必须是以终结符开头\\
       一般用来定义一个单词
\end{itemize}
4种类型的文法约束依次为:左侧不能是句子,左侧长度小于右侧,左侧只有非终结符,右侧以终结符开头

\begin{figure}[H]
 \centering
 \includegraphics[scale = 0.3]{assets/CompilerConstructionPrinciples-part1_f8a53.png}
 \caption{四种文法之间的包含关系}
\end{figure}

\section{上下文无关文法及其语法树}
上下文无关文法:2型文法,有足够的能力描述现今程序设计语言的语法结构

例子:算数表达式:
$E\to i|E+E|E*E|(E)$\\
$<\text{赋值语句}> \to i := E$\\
$<\text{条件语句}> \to if <\text{条件}> then <\text{语句}>$\\
\makebox[2cm]{}$| if <\text{条件}> then <\text{语句}> else <\text{语句}>$

\subsection{规范推导和规范句型}
\textbf{最左/右 推导}:\\
在推导任何一步$\alpha\to \beta$时,其中$\alpha,\beta$为句型,都是对$\alpha$中的最左/右的
非终结符进行替换,则成为最左/右推导。

{\color{blue}推导到底是多次直接推导,还是多条产生式?前者}

\textbf{规范推导:}最右推导\\
\textbf{规范规约:}最左规约\\
\textbf{规范句型:}由规范推导所得的句型

\textbf{语法树:}推导的过程,可以展开成语法树,但是从语法树是看不出语法树的构造顺序的(即推导的过程)\\
在语法树上,从左到右读取叶子节点,可以得到句型或句子。

\textbf{语法树的定义:}满足下面几个条件的树,则称为文法G的语法树。
\begin{itemize}
 \item 每个节点都有标记,是终结符或者非终结符中一个符号
 \item 根节点的标记是起始符(S)
 \item 若一个节点至少有一个除它自己之外的子孙,则这个节点的标记一定是非终结符
 \item 语法树得到的一定是文法G的产生式
\end{itemize}

\textbf{文法的二义性:}\\
一个文法存在某个句子对应2个不同的语法树,则称文法有二义性。

对于程序设计语言来说,希望它的文法是无二义性的,因为希望对它的每个语句的分析是唯一的。

因此,如果出现了有二义性的文法,可以尝试把它转换成等价的无二义性的文法。

\subsection{句型的分析}
\textbf{句型的分析:}识别给定串是否是某文法的句型。

句型的分析主要有两种方法:
\begin{itemize}
 \item 自上而下的分析法(推导)
 \item 自下而上的分析法(归约)
\end{itemize}

\textbf{自上而下的算法:}从开始符出发,反复使用文法中的产生式,
寻找匹配的推导。(从根节点开始不断构造语法树,找出与给定串匹配的语法树)\\
这个算法的关键之处是\textbf{如何选择产生式?}(剧透:LL(1)文法只有一个待选择的产生式 , 多个选择的时候可以使用回溯法(即多个可能都进行搜索尝试))

注:这里生成语法树的是为后面的语法分析和词法分析所用,对于句型分析(词法分析)是没有作用的。

这里的描述的意思是:如果能正确构造出这个串的语法树那么它就是合法的句子(而构造的过程则是不断推导的过程,通过推导来确定语法树的创建过程,而不是由语法树来决定推导的过程)

语法树的末端结点符号串正好是输入符号串。

\spaceline
\textbf{自下而上的算法:}从给定串出发,归约到起始符(从叶子节点出发,向上构造语法树)\\
这个算法的关键之处是\textbf{如何确定可归约串?}(剧透:通过构造出语法各个状态的转移过程来确定归约串,当输入归约串的同时,状态也不断发生变化,而下一个动作(状态转移还是归约)则由当前的输入 和 当前的状态决定 , 问题转变成 \textbf{如何构造这个分析表?})

\sout{自上而下的话,感觉就是凭空想象,直到推出想要的结果,而自下而上的话,则是从句子本身出发,带有明确的方向性,更为简单。}胡说八道,虽然自上而下只知道回溯法

\textbf{归约的描述:}$abc|-S$,表示串$abc$被归约成$A$

这里的语法树也是和上面一样,是随着归约的过程构造,而不是由语法树来进行归约。

\spaceline
要确定可归约串,首先要了解\textbf{短语,直接短语,句子}的概念。{\color{red}为什么?\sout{回头需要去看算法的原理},辅助分析建立的概念}\\
\textbf{短语:}若$\renewcommand{\arraystretch}{0.5}S\begin{array}{c} * \\ \Rightarrow \end{array} \alpha A\delta
 \begin{array}{c} * \\ \Rightarrow \end{array} \beta$,则称$\beta$是句型{\color{red}$\alpha \beta \delta$}相对于$A$的短语。
(从语法树上来看,短语就是语法树的每棵子树的叶子组成的句子 )

{\color{blue}到底是句型$\alpha A \delta$ 还是 句型$\alpha \beta \delta$?后者}

直接短语:若$\renewcommand{\arraystretch}{0.5}S\begin{array}{c} * \\ \Rightarrow \end{array} \alpha A\delta
 {\color{red}\Rightarrow}  \beta$,则称$\beta$是句型{\color{red}$\alpha \beta \delta$}相对于$A$的直接短语。
(从语法树上来看,每个叶子都是直接短语)

\textbf{句柄:}一个句型的最左直接短语为句柄(从语法树上来看,最左的叶子就是句柄)\\
\textbf{可归约串}:句柄即可归约串。

那么自下而上的算法则是,每次选取语法树的句柄归约。

这里为方便理解,从语法树的角度来短语,直接短语,句柄这些概念,但实际并不是从语法树来确定可归约串,而是从句柄在语法树的位置,我们能更加清晰的明白为什么在这个句柄的情况下,为什么要使用这个句型进行归约。

后面建立分析表的时候,实际上就是枚举出了所有句型的语法树(状态转移图),从图上得到是归约还是继续状态转移(未到语法树的叶子) , 从而确定每个句型的句柄。当多个句型有相同的句柄的时候,就出现了冲突,就需要结合其他技术来解决问题(LR(1) , SLR(0) ,LALR文法等)

{\color{red} 后面补上一个求直接短语,短语, 句柄以及归约的例子}

\spaceline
\textbf{多余规则:}
\begin{itemize}
 \item 不可到达
 \item 不可终止
\end{itemize}

\section{词法分析}
\textbf{单词描述的工具:}正规文法(3型文法)和正则式(正则表达式)

\subsection{正规文法与正则式}
\textbf{正规文法}也称三星文法$G=(V_N , V_T , S , P)$,其P中的一条规则都有以下形式:$A\to aB$或$A \to a$,其中$A,B\in V_N , a \in V_T^*$。\\
即(一句话概括)文法的产生式的右侧以终结符开头。

\spaceline
\textbf{正则式:}也称正则表达式。

既然有了正规文法,为什么还需要正则式?(作用)\\
因为正则式可以直观地看出单词的构成

\spaceline
给定集合$\Sigma$
\begin{itemize}
 \item $\epsilon$和$\Phi$\footnote{$\Phi = \{\epsilon\}$}都是某个集合$\Sigma$上的正规式。
 \item 集合内的任意元素都是集合$\Sigma$的正规式
 \item 正规式经过以下运算之后,还是集合$\Sigma$的正规式
       \begin{itemize}
        \item "$|$" 或
        \item "$.$" 连接
        \item "$*$" 闭包
        \item 优先级 $* > . > |$
       \end{itemize}
\end{itemize}

例子:令$\Sigma = \{a,b\}$ , $\Sigma$上的正规式和相应的正规集的例子如下:
\begin{table}[H]
 \centering
 \begin{tabular}{c|c}
  \hline
  正规式       & 正规集                                                \\ \hline
  $a$          & $\{a\}$                                               \\
  $a|b$        & $\{a,b\}$                                             \\
  $ab$         & $\{ab\}$                                              \\
  $(a|b)(a|b)$ & $\{aa,ab,ba,bb\}$                                     \\
  $a^*$        & $\{\epsilon , a, aa,\cdots\}$,即任意个a的串           \\
  $(a|b)^*$    & $\{\epsilon , a, b ,aa ,ab\cdots\}$,即所有a,b组成的串 \\ \hline
 \end{tabular}
\end{table}

\subsection{正规文法和正则式的等价性}
\textbf{将正规式转换成正规文法:}\\
选择一个非终结符$S$生成类似产生式的形式:$S\to r$ ,并将$S$定为$G$的识别符号。\\

具体转换规则如下图,原产生式右边为正规式

\begin{figure}[H]
 \centering
 \includegraphics[scale = 0.3]{assets/CompilerConstructionPrinciples-part1_59a3f.png}
 \caption{正规式转换成正规文法}
\end{figure}

\textbf{例子}:将$r = a(a|d)^*$转换成相应的正规文法。{\color{red}\sout{书本上貌似写多了,以下为自己观点},PPT是下面的写法}
\begin{itemize}
 \item
       $S\to a(a|d)^*$
 \item
       $S\to aA$\\
       $A \to (a|d)^*$
 \item
       $S \to aA$\\
       $A \to (a|d)A$  \\
       $A \to \epsilon$
 \item
       $S \to aA$\\
       $A \to aA$\\
       $A \to dA$\\
       $A \to \epsilon$
\end{itemize}

\textbf{将正规文法转换成正规式:}emmm,这个转换看直觉吧,懒得写了

{\color{red}为什么最右推导,最左规约是合理的?有什么直观的理解?}

\subsection{有穷自动机}
\textbf{有穷自动机}:也称有限自动机,是一种识别装置,能准确识别正规集,即识别正规文法所定义的语言和正规式所表示的集合。\\
有穷自动机本质上是一个状态转移图。

\spaceline
\subsubsection{确定的有穷自动机(DFA)}
定义:一个确定的有穷自动机$M$是一个五元组
\[M = (K , \Sigma , f , s , z)\]
其中,
\begin{itemize}
 \item [(1)] K是一个有穷集,它的每个元素称为一个状态
 \item [(2)] $\Sigma$是一个有穷字母表,它的每个元素称为一个输入符号,所以也称为$\Sigma$为输入符号表。
 \item [(3)] f是转换函数, 是$K\times \Sigma \to K$上的映像。
 \item [(4)] $S\in K$,是唯一的一个初态
 \item [(5)] $Z \subseteq  K$,是一个终态集,终态也称为可接受状态或结束状态。
\end{itemize}
确定性体现在:
\begin{itemize}
 \item 初态是唯一的
 \item 每个状态对应的唯一的下一个状态
\end{itemize}

\textbf{DFA可以使用状态图或状态矩阵表示:}图\ref{fig1}和图\ref{fig2}分别表示状态图和状态矩阵。

\begin{figure}[H]
 \centering
 \includegraphics[scale = 0.3]{assets/CompilerConstructionPrinciples_5398e.png}
 \caption{状态转移图:初态结点冠以"$\rightarrow$"或标以"$-$",终态结点用双圈表示或标以"$+$"}
 \label{fig1}
\end{figure}

\begin{figure}[H]
 \centering
 \includegraphics[scale = 0.3]{assets/CompilerConstructionPrinciples_c1ff8.png}
 \caption{状态转移矩阵:可以用$\rightarrow$标明初态;否则第一行即初态,相应终态行在表的右端标以1,非状态标以0}
 \label{fig2}
\end{figure}

\subsubsection{不确定的有穷自动机(NFA)}
定义:一个不确定的有穷自动机$M$是一个五元组
\[M = (K , \Sigma , f , s , z)\]
其中,
\begin{itemize}
 \item [(1)] K是一个有穷集,它的每个元素称为一个状态
 \item [(2)] $\Sigma$是一个有穷字母表,它的每个元素称为一个输入符号,所以也称为$\Sigma$为输入符号表。
 \item [(3)] f是转换函数, 是$K\times {\color{red}\Sigma^*}$到$K$的全体子集的映像,即$K\times {\color{red}\Sigma^*} \to 2^K$,其中$2^K$表示$K$的幂集
       \footnote{$K\times  \Sigma^*$指的是两者的笛卡尔积,表示一个状态与每个输入的组合 ,
        然后这里取闭包$\Sigma^*$,表示每个状态与多个可能输入的组合(也可以分解成一个状态一个输入,这样写应该是为了简化,因为画图的时候,采取前者而不是逐个画)}
       \footnote{所谓幂集(Power Set), 就是原集合中所有的子集(包括全集和空集)构成的集族}
       {\color{blue}这一段是什么鬼?见脚注}
 \item [(4)] $S\subseteq K$,是一个非空初态集
 \item [(5)] $Z \subseteq  K$,是一个终态集
\end{itemize}
确定性体现在:
\begin{itemize}
 \item 初态不唯一(有初态集)
 \item 每个状态对应的在同一个输入,有多个可能的状态转移。(即映射不唯一,一个状态映射到所有状态的某个子集)
\end{itemize}

\subsubsection{NFA转换为等价的DFA}
\textbf{定理:}设$L$为一个由NFA接受的集合,则存在一个接受$L$的DFA。

\spaceline
\textbf{子集法:}将NFA转换成接受同样语言的DFA\\
为一个NFA构造相应的DFA的基本想法是让DFA的每一个状态对应NFA的一组状态。(即把可能转移的多个不确定的状态的组合表示成一个状态来看)
具体算法见\ref{fig-zijifa}{\color{blue}在进行子集法的时候,也会有无用状态去掉}

\begin{itemize}
 \item $\epsilon$合并\\
       经过$\epsilon$弧得到的状态可以合并成一个状态
 \item 状态合并\\
       当前状态所能直接转移到的所有状态合并成一个状态
 \item 状态集合I的$\epsilon$闭包\\
       $\epsilon-closure(I)$状态集I中的任何状态S经过任一条$\epsilon$弧而能到达的状态
 \item 状态集合I的$a$弧转换\\
       $move(I,a)$,所有那些可从$I$中的某个状态经过一条a弧而达到的状态的全体。
\end{itemize}

\begin{figure}[H]
 \centering
 \includegraphics[scale = 0.1]{assets/CompilerConstructionPrinciples_763bd.png}
 \caption{子集构造算法}
 \label{fig-zijifa}
\end{figure}

\begin{itemize}
 \item C为DFA的状态集合,初始元素为起始状态的空闭包
 \item while所做的就是不断以C中的状态转移出新的状态加入C中,直到所有的装备都被使用过
 \item 所谓的转移,指的就是$\epsilon-closure(Move(T,a))$\\
       其中,$T$为C中的一个状态(它的值为NFA中某些状态的集合) , $a$表示经过的弧, 结果可以简记为$I_a$
 \item 终态的确定\\
       上面的算法确定的NFA对应到DFA有哪些状态,但是还未确定终态。\\
       DFA的\textbf{终态}就是DFA中和NFA中的终态有交集的状态。
\end{itemize}

\subsubsection{DFA的化简}
一个DFA可以通过\textbf{消除无用状态}和\textbf{合并等价状态}来转换成一个等价的最小状态的DFA。\\
也就是化简主要分两个部分。

\spaceline
所谓最简的DFA指的是它没有多余的状态,并且它的状态中没有两个是互相等价的。

\spaceline
\textbf{无用状态:}
\begin{itemize}
 \item 从该自动机的开始状态出发,任何输入也不能到达的那个状态
 \item 从这个状态没有通路到达终态
\end{itemize}
{\color{blue}有什么算法能解决上面这个问题呢?从起点开始DFS,在能到达终点的前提下,所能访问到的所有的点都是有效的。其他都是无效的。\
注:在进行子集法的时候,也会有无用状态去掉}

\spaceline
\textbf{等价状态:}
\begin{itemize}
 \item [(1)] 一致性条件\\
       状态s和t的必须同时为可接受状态或不可接受状态。\footnote{可接受状态指的是终态,不可接受状态指的是非终态。}
 \item [(2)] 蔓延性\\
       对于所有输入符号,状态s和t必须转换到等价的状态里。
\end{itemize}

\spaceline
那么如何找出等价的状态进而化简呢?\\
\textbf{分割法:}把一个DFA(不含多余状态)的状态分成一些不相交的子集,使得任何不同的两个子集的状态都是可区别的,而
同一子集中的任何两个状态都是等价的。

\begin{itemize}
 \item 终态与非终态能初始划分出初始的两个不等价的子集。
 \item 对于每个集合,尝试所有的输入,如果集合内的两个状态不能转移到相同集合,那么这两个状态是不等价的,需要把这两个状态拆分到两个新的子集内。
 \item 尝试所有的输入可能,最终得到的集合内的元素就是相互等价的元素\\
       (因为尝试的所有的输入都是转移到相同的状态(可蔓延性) , 而一致性在初始的时候已经明确)。
\end{itemize}

\spaceline
\textbf{例子:}把图\ref{example-zijifafengefa}表示的NFA确定化和最小化。
\begin{figure}[H]
 \centering
 \begin{tikzpicture}
  [every initial by arrow/.style={double distance = 3,-Implies}]
  \node[state,initial,initial text=,double]  (A0)  {0};
  \node[state]  (A1)  [right=of A0] {1};

  \path [->] (A0) edge [bend left]  node [above] {a,b} (A1)
  edge [loop above] node {a} (A0)
  (A1) edge [bend left] node [above] {a} (A0);
 \end{tikzpicture}
 \caption{例子:子集法、分割法例子}
 \label{example-zijifafengefa}
\end{figure}

\begin{itemize}
 \item 确定化,见表\ref{example-zijifa}
       \begin{table}[H]
        \centering
        \begin{tabular}{c|c|c}
         \hline
         set         & input a      & input b    \\  \hline
         $A=\{0\}$   & $\{0,1\}(B)$ & $\{1\}(C)$ \\
         $B=\{0,1\}$ & $\{0,1\}(B)$ & $\{1\}(C)$ \\
         $C=\{1\}$   & $\{1\}(C)$   & $\Phi$     \\
         \hline
        \end{tabular}
        \caption{例子:子集法 表}
        \label{example-zijifa}
       \end{table}

       对应状态图为:见图\ref{fig-example-zijifafengefa}
       \begin{figure}[H]
        \centering
        \begin{tikzpicture}[node distance=60,
          every initial by arrow/.style={double distance = 3,-Implies}]
         \node[state,initial,initial text=,double]  (A)  {A};
         \node[state,double]  (B)  [ right  of=A] {B};
         \node[state]  (C)  [below right  of=A] {C};

         \path [->] (A) edge [bend left]  node [above] {a} (B)
         edge [bend left]  node [above] {b} (C)
         (B) edge [loop above]  node [above] {a} (B)
         edge [bend left]  node [below] {b} (C)
         (C) edge [bend left]  node [above] {a} (A);
        \end{tikzpicture}

        \caption{例子:子集法 图}
        \label{fig-example-zijifafengefa}
       \end{figure}
 \item 最小化,见表\ref{example-fengefa}
       \begin{table}[H]
        \centering
        \begin{tabular}{c|c}
         \hline
         input & sets              \\  \hline
         init  & $\{C\}$,$\{A,B\}$ \\
         a     & $\{C\}$,$\{A,B\}$ \\
         b     & $\{C\}$,$\{A,B\}$ \\
         \hline
        \end{tabular}
        \caption{例子:分割法 表}
        \label{example-fengefa}
       \end{table}

       因此$A,B$两个状态是等价的。
       对应的状态图为\ref{fig-example-fengefa}
       \begin{figure}[H]
        \centering
        \begin{tikzpicture}
         [every initial by arrow/.style={double distance = 3,-Implies}]
         \node[state,initial,initial text=,double]  (A0)  {0};
         \node[state]  (A1)  [right=of A0] {1};

         \path [->] (A0) edge [bend left]  node [above] {b} (A1)
         edge [loop above] node {a} (A0)
         (A1) edge [bend left] node [above] {a} (A0);
        \end{tikzpicture}
        \caption{例子:分割法 图}
        \label{fig-example-fengefa}
       \end{figure}

       另外,可以观察出,NFA的图\ref{example-zijifafengefa}和最小化后的DFA的图\ref{fig-example-fengefa}
       只是相差的一个$a$而已,实际上他们的确是等价的。\\
       因为  \begin{tikzpicture}
        [every initial by arrow/.style={double distance = 3,-Implies}]
        \node[state,initial,initial text=,double]  (A0)  {0};
        \path [->] (A0) edge [loop above] node {a} (A0);
       \end{tikzpicture}
       已经足够描述任意长度的a串,而无需\begin{tikzpicture}
        [every initial by arrow/.style={double distance = 3,-Implies}]
        \node[state,initial,initial text=,double]  (A0)  {0};
        \node[state]  (A1)  [right=of A0] {1};

        \path [->] (A0) edge [bend left]  node [above] {a} (A1)
        (A1) edge [bend left] node [above] {a} (A0);
       \end{tikzpicture}
\end{itemize}

\subsubsection{正规式与有穷自动机的等价性}
正规式和有穷自动机的等价性由以下两点说明:
\begin{itemize}
 \item 对于字母表(输入表)$\Sigma$上的NFA M,可以构造一个$\Sigma$上的正规式r,使得$L(r) = L(M)$
 \item 对于$\Sigma$上的每个正规式r,可以构造一个$\Sigma$上的NFA M,使得$L(r) = L(M)$
\end{itemize}

\textbf{正规式$\to$NFA}:分两个步骤
\begin{itemize}
 \item 建立x节点,使用$\epsilon$连接到初态,建立y节点,使用$\epsilon$弧连接终态
 \item 不断消去弧,直到只剩下x,y两个节点
\end{itemize}
消去规则:
\begin{itemize}
 \item \begin{tikzpicture}
        [every initial by arrow/.style={double distance = 3,-Implies}]
        \node[state]  (A0)  {0};
        \node[state]  (A1)  [right=of A0] {1};
        \node[state]  (A2)  [right=of A1] {2};
        \path [->] (A0) edge  node [above] {$r_1$} (A1);
        \path [->] (A1) edge  node [above] {$r_2$} (A2);
       \end{tikzpicture} $\to$ \begin{tikzpicture}
        [every initial by arrow/.style={double distance = 3,-Implies}]
        \node[state]  (A0)  {0};
        \node[state]  (A2) [right=of A0] {2};
        \path [->] (A0) edge  node [above] {$r_1r_2$} (A2);
       \end{tikzpicture}

 \item \begin{tikzpicture}
        [every initial by arrow/.style={double distance = 3,-Implies}]
        \node[state]  (A0)  {0};
        \node[state]  (A1) [right=of A0] {1};
        \path [->] (A0) edge[bend left]  node [above] {$r_1$} (A1);
        \path [->] (A0) edge[bend right]  node[above]  {$r_2$} (A1);
       \end{tikzpicture} $\to$ \begin{tikzpicture}
        [every initial by arrow/.style={double distance = 3,-Implies}]
        \node[state]  (A0)  {0};
        \node[state]  (A1) [right=of A0] {1};
        \path [->] (A0) edge  node[above]  {$r_1|r_2$} (A1);
       \end{tikzpicture}

 \item \begin{tikzpicture}
        [every initial by arrow/.style={double distance = 3,-Implies}]
        \node[state]  (A0)  {0};
        \node[state]  (A1) [right=of A0] {1};
        \node[state]  (A2) [right=of A1] {2};
        \path [->] (A0) edge  node[above]  {$r_1$} (A1);
        \path [->] (A1) edge  node [above] {$r_3$} (A2);
        \path [->] (A1) edge [loop above] node [above] {$r_2$} (A1);
       \end{tikzpicture} $\to$ \begin{tikzpicture}
        [every initial by arrow/.style={double distance = 3,-Implies}]
        \node[state]  (A0)  {0};
        \node[state]  (A1) [right=of A0] {1};
        \path [->] (A0) edge  node[above]  {$r_1r_2^*r_3$} (A1);
       \end{tikzpicture}
\end{itemize}

\textbf{NFA$\to$ 正则式}:
\begin{itemize}
 \item 环对应闭包\\
       注意,闭包是包括空的
 \item 连续输入对应与
 \item 分支对应或
\end{itemize}

\subsubsection{正规文法和有穷自动机的等价性}
主要有以下几点性质:
\begin{itemize}
 \item 弧的输入(字母表)就是终结符
 \item 非终结符就是状态
 \item 新增一个状态Z作为终态
 \item 对于形如$A\to tB$的产生式,得到$f(A,t) = B$的转换函数。即\begin{tikzpicture}
        [every initial by arrow/.style={double distance = 3,-Implies}]
        \node[state]  (A0)  {A};
        \node[state]  (A1) [right=of A0] {B};
        \path [->] (A0) edge  node[above]  {t} (A1);
       \end{tikzpicture}

 \item 对于形如$A\to t$的产生式,得到$f(A,t) = Z$的转换函数,即\begin{tikzpicture}
        [every initial by arrow/.style={double distance = 3,-Implies}]
        \node[state]  (A0)  {A};
        \node[state,double]  (A1) [right=of A0] {Z};
        \path [->] (A0) edge  node[above]  {t} (A1);
       \end{tikzpicture}
\end{itemize}

\textbf{正则式,正规文法,有穷自动机的关系?}\\
正则式和正规文法是描述工具,有穷自动机是识别工具,三者可等价相互转换。

\subsection{词法分析程序的构造}
词法分析程序的构造主要分为4个步骤:
\begin{itemize}
 \item 接口类型的确定\\
       语法分析程序可以有种利用方式:\\
       一是作为单独的子程序,其结果作为语法分析程序的输入。\\
       二是作为语法分析程序的一部分
 \item 确定单词的分类和Token串结构\\
       单词常见的类型有:关键字,标识符等\\
       token串一般为2元组(单词的种别,单词自身的值)  (需要自己对种类进行编号)

 \item 特殊问题
       \begin{itemize}
        \item 标识符和保留字的区分\\
              事先构造好保留字表,在拼出标识符的单词下能查询保留字表,如果是保留字,则作为保留字处理,否则才是标识符
        \item 空格、制表符及换行符的处理\\
              无用空格和制表符要删除\\
              字符串内的空格不能删\\
              换行符不能删,对于错误处理起作用
        \item 复合型Token的处理\\
              比如"!=" ,读取到"!"的时候,还必须读取下一个字符进行判断
        \item 括号,引号等成对符号的匹配问题\\
              这一个本应是语法分析的工作,但是可以词法分析程序中完成\\
              处理方式是,设置计数器,每遇到左括号计数器+1,右括号计数器-1,最终结果不为0,说明括号不匹配
       \end{itemize}

 \item 使用状态转换图构造语法分析程序\\
       文法or正规式$\to$ NFA $\to$ DFA $\to$ 最简DFA
\end{itemize}

{\color{red}后面补上正规式,正规文法与有穷自动机相互转换的例子}
