\documentclass[UTF8,a4paper]{ctexart}
\usepackage[margin=1in]{geometry}
\usepackage{fancyhdr,hyperref}
\pagestyle{fancy}
\hypersetup{hidelinks}

\lhead{\bfseries \leftmark}
\chead{}
\rhead{SCUT}
\lfoot{\url{https://github.com/285571052}}
\cfoot{qhy}
\rfoot{\thepage}
\setlength{\headheight}{13pt}
\renewcommand{\headrulewidth}{0.4pt}
\renewcommand{\footrulewidth}{0.4pt}

\setlength{\parindent}{0pt}
\newcommand{\spaceline}{\vspace{\baselineskip}}

\author{ qhy }
\date{\today}
\title{高性能计算与云计算}

\begin{document}
  \maketitle
  \tableofcontents
  \newpage

  \section{介绍}

  \textbf{高性能与云计算关注的问题:}\\
  如何使计算机更快更方便地解决更大规模的问题。

  \spaceline
  \textbf{做得更快的方法:}
  \begin{enumerate}
    \item [1.] work harder

    换一台更好的计算机

    \item [2.] work smarter

    修改为更加优化的方法

    \item [3.] getting help

    并行处理
  \end{enumerate}

  单核CPU的提升遇到瓶颈,提高计算机性能需要另寻出路:多核时代,由AMD首先推出。

  \spaceline
  \textbf{多核运算的速度一定比单核的CPU快吗?}\\
  不一定,考虑到散热等问题,多核CPU的主频往往比单核的低,若程序只能发挥出一个内核效用的话,
  自然不如单核CPU快。\\
  想要发挥多核功能,设计的软件首先要做并行运算。\\
  多核的出现,使得高性能计算进入普及时代。

  \spaceline
  \textbf{大规模数据处理当前面临的问题:}
  \begin{enumerate}
    \item [1.] 大规模PC集群可靠性差

    一台PC的MTBF(Mean Time Between Failure) = 3年

    则1000台PC的MTBF = 1天

    商用网络 = 低带宽

    这个要求运行系统具备良好的可扩展性、良好的容错能力

    \item [2.] 并行/分布式程序开发、调试困难

    \begin{enumerate}
      \item 数据划分
      \item 任务调度
      \item 任务之间的通信
      \item 错误的处理、容错
    \end{enumerate}

    这要求编程模型具备一定表达能力、很好的简单易用性
  \end{enumerate}

  大数据时代的高性能计算:云计算:通过互联网将资源按需服务的形式提供给用户。

  \spaceline
  \textbf{课程内容}
  \begin{enumerate}
    \item 高性能计算系统与其结构模型
    \item 并行算法设计
    \item 并行程序的设计原理和方法
    \item 云计算
    \item 高性能计算与云计算的应用及发展趋势
  \end{enumerate}

  \spaceline
  \textbf{1高性能计算与云计算的3大基础:}
  \begin{enumerate}
    \item 计算

    数据处理能力

    \item 存储

    \item 通信
  \end{enumerate}

  \spaceline
  \textbf{高性能计算与分布式计算的区别?没仔细记,以下瞎写}\\
  高性能计算是把计算任务分配到各个节点上计算\\
  分布式计算则是把各种资源分布在各个节点上计算

  \spaceline
  \textbf{高性能计算的模型框架???没记录,名字也不确定对不对}
\end{document}
